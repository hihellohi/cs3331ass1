\documentclass[a4paper]{article}

\usepackage[english]{babel}
\usepackage[UKenglish]{isodate}
\usepackage[parfill]{parskip}
\usepackage{float}
\usepackage{graphicx}
\usepackage{booktabs}
\usepackage{csquotes}
\usepackage{listings}
\usepackage[margin=2.5cm]{geometry}
\usepackage{enumerate}
\usepackage{siunitx}

\newcommand{\nth}{\textsuperscript}

\title{Assignment 1 Report}
\author{Kevin Zihong Ni\\\texttt{z5025098}}

\begin{document}

\maketitle

\section{Design Overview}

Both the sender and the receiver of my implementation of STP was written in C++ (as authorised by the lecturer Apro S. Kanhere
over email). I chose to do this as C++, unlike C offers the standard template library, allowing the use of data structures such as
maps and queues which would have required significant amounts of time investment to implement on my own. The code does 
\textbf{not} require a C++11 compiler.

\subsection{Sender}
The sender keeps a queue of segments that have been sent but have yet to be acked. This queue is never greater than the maximum
window size and is "replenished" from the input file whenever its size falls below the maximum (unless the EOF has been reached).
Whenever an ack is received, segments are popped off the front of the queue until the sequence number of the
next packet is no longer greater than the ack number of the received acknowledgement. If the ack number is equal to the segment
at the front of the queue (before any popping), the duplicate ack counter is incremented. When this counter reaches 3, or if the
timeout period expires, the packet at the front of the queue is retransmitted.

\subsection{Receiver}
The receiver continuously listens for incoming data on the given socket. It keeps track of the lowest sequence number that it has
yet to receive. An acknowledgment packet with this number is sent every time a data packet is received.
If a received packet has sequence number less than this number, it just throws it away as it already has that
data, making it a duplicate. If it is greater, then it caches the packet into a map, indexing it by its sequence number.
if it is equal to the number, then it increments the number by the length of the received packet and searches the map
for a packet that has a sequence number equal to the new number. It repeats this until no such packet exists in the map.

\subsection{Extension}
The delay module of the extension was implemented with the use of a min priority queue. Whenever a packet was delayed, it was 
pushed onto the priority queue with the current time in milliseconds plus a random constant. Whenever the current time was greater
than the priority of the next item of the priority queue, the item was sent and popped off the priority queue.

The estimation module of the extension was implemented using a map. Whenever a segment was "sent", its sequence number and a 
timestamp was recorded into a map. If a retransmission occured, a flag was set that would only be reset on a successful ack.
If the flag was not set, then the current time was used to recalculate the estimations whenever an acknowledgment was received.

\subsection{Features}
The list of successfully implemented features is as follows:
\begin{itemize}
	\item Successful reliable transmission of a file.
	\item Three way handshake (SYN, SYNACK, ACK) to establish connection. It is assumed that this is always received by the
		receiver (as per the specifications). This \textbf{can} be delayed.
	\item Randomized starting sequence number. Can handle integer overflows. Greater than and less than defined by window size.
	\item Four way termination (FIN, ACK, FIN, ACK). The middle two packets were combined into one (as per the example given in
		the specifications). Like the handshake packets, these may be delayed but never dropped.
	\item Single timer for timeout operation for sender.
	\item Fast retransmit after 3 duplicate acks.
	\item Both sequence number and acknowledgement number included in header.
	\item Sender takes in maximum segment size and maximum window size as input.
	\item PLD module randomly drops and delays packets with chances and maximum delay given as input.
	\item Dynamic timeout calculated from sample RTTs. (This has a floor of 1ms because 0ms resulted in timeouts happening
		constantly resulting in significant amounts of natural packet loss due to the sheer volume of UDP datagrams being sent)
\end{itemize}

\section{STP Header}
\lstinputlisting[language=C, caption=STP Header Struct Definition]{header.h}
\begin{table}[H]
\centering
\caption{STP Header Field Descriptions}
\label{header}
\begin{tabular}{@{}llp{0.8\textwidth}@{}}
\toprule
Field & Bytes & Description \\ \midrule
\verb|n_seq| & 4 & Contains the sequence number  of the segment. Byte-stream oriented, meaning that the sequence number reflects the index of the first data byte instead of the index of the packet. \\
\verb|n_ack| & 4 & Contains the acknowledgement number when sending an acknowledgement packet. Is also used to communicate the maximum segment size during the handshake. \\
\verb|len| & 4 & The length of the data packet (excluding header) in bytes. \\
\verb|size| & 4 & Only used during handshake. Contains the size of the window in bytes (necessary to handle sequence number overflows). \\
\verb|flags| & 1 & Indicates whether the segment is a SYN, FIN, ACK and/or DATA segment. \\ \bottomrule
\end{tabular}
\end{table}

\section{Extension Questions}

\end{document}
